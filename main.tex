% Arquivo LaTeX de exemplo de dissertação/tese a ser apresentados à CPG do IME-USP
% 
% Versão 5: Sex Mar  9 18:05:40 BRT 2012
%
% Criação: Jesús P. Mena-Chalco
% Revisão: Fabio Kon e Paulo Feofiloff
%  
% Obs: Leia previamente o texto do arquivo README.txt

\documentclass[12pt,oneside,a4paper]{book}

% ---------------------------------------------------------------------------- %
% Pacotes
\usepackage[OT1]{fontenc}
\usepackage[spanish]{babel}
\usepackage[utf8]{inputenc}
\usepackage{textcomp}
\usepackage[pdftex]{graphicx}           % usamos arquivos pdf/png como figuras
\usepackage{setspace}                   % espaçamento flexível
\usepackage{indentfirst}                % indentação do primeiro parágrafo
\usepackage{makeidx}                    % índice remissivo
\usepackage[nottoc]{tocbibind}          % acrescentamos a bibliografia/indice/conteudo no Table of Contents
\usepackage{courier}                    % usa o Adobe Courier no lugar de Computer Modern Typewriter
\usepackage{type1cm}                    % fontes realmente escaláveis
\usepackage{listings}                   % para formatar código-fonte (ex. em Java)
\usepackage{titletoc}
%\usepackage[bf,small,compact]{titlesec} % cabeçalhos dos títulos: menores e compactos
\usepackage{titlesec}
\usepackage[fixlanguage]{babelbib}
\usepackage[font=small,format=plain,labelfont=bf,up,textfont=it,up]{caption}
\usepackage[usenames,svgnames,dvipsnames]{xcolor}
\usepackage[a4paper,top=2.54cm,bottom=2.0cm,left=2.0cm,right=2.54cm]{geometry} % margens
%\usepackage[pdftex,plainpages=false,pdfpagelabels,pagebackref,colorlinks=true,citecolor=black,linkcolor=black,urlcolor=black,filecolor=black,bookmarksopen=true]{hyperref} % links em preto
\usepackage[pdftex,plainpages=false,pdfpagelabels,pagebackref,colorlinks=true,citecolor=DarkGreen,linkcolor=NavyBlue,urlcolor=DarkRed,filecolor=green,bookmarksopen=true]{hyperref} % links coloridos
\usepackage[all]{hypcap}                    % soluciona o problema com o hyperref e capitulos
\usepackage[square,sort,nonamebreak]{natbib} % citação bibliográfica textual(plainnat-ime.bst)
\bibpunct{[}{]}{;}{a}{\hspace{-0.7ex},}{,} % estilo de citação. Veja alguns exemplos em http://merkel.zoneo.net/Latex/natbib.php

\fontsize{60}{62}\usefont{OT1}{cmr}{m}{n}{\selectfont}

%%%%%----MIO
\usepackage{blindtext} %lorem-ipsum
\usepackage{float}
\usepackage{array,multirow}
%\usepackage{enumerate}
%\usepackage{tikz}
\usepackage{subfig}
\usepackage{amsmath}
\usepackage{amsthm}
\usepackage{amsfonts}
\usepackage{amssymb}
%\theoremstyle{plain}
%Teoremas, definiciones y ejemplos
\newtheorem{thm}{Teorema}[section] % reset theorem numbering for each chapter
\theoremstyle{definition}
\newtheorem{defn}[thm]{Definición} % definition numbers are dependent on theorem numbers
\newtheorem{exmp}[thm]{Example} % same for example numbers
%\renewcommand{\figurename}{Figura.}
%\renewcommand{\tablename}{Tabla.}
\addto\captionsenglish{\renewcommand{\figurename}{Figura.}}
\addto\captionsenglish{\renewcommand{\tablename}{Tabla.}}
\usepackage{wrapfig}
%%%%%---MIO
%%% tabla
%\usepackage[table]{xcolor}
%\usepackage[margin=1in]{geometry}
\usepackage{tabularx}
\usepackage{enumitem}
\setlist{nolistsep}
\definecolor{green}{HTML}{66FF66}
\definecolor{myGreen}{HTML}{009900}
\usepackage{appendix}

%\renewcommand{\familydefault}{\sfdefault}
\renewcommand{\arraystretch}{1.5}

\renewcommand\spanishtablename{Tabla}
\usepackage{csvsimple}
\usepackage{pdfpages}

\titleformat{\chapter}[display]
{\normalfont\huge\bfseries}{\chaptertitlename\ \thechapter}{2pt}{\Large\MakeUppercase}
  
\titleformat{\section}{\normalfont\scshape\bfseries}{\textbf{\thesection}}{1em}{\MakeUppercase}
\titleformat{\subsection}{\normalfont\bfseries}{\textbf{\thesubsection}}{1em}{}
%\allsectionsfont{\mdseries\scshape}


\usepackage{hyperref}

%%%
% ---------------------------------------------------------------------------- %
% Cabeçalhos similares ao TAOCP de Donald E. Knuth
\usepackage{fancyhdr}
\pagestyle{fancy}
\fancyhf{}
\renewcommand{\chaptermark}[1]{\markboth{\MakeUppercase{#1}}{}}
\renewcommand{\sectionmark}[1]{\markright{\MakeUppercase{#1}}{}}
%\renewcommand{\chaptermark}[1]{\markboth{{#1}}{}}
%\renewcommand{\sectionmark}[1]{\markright{{#1}}{}}
\renewcommand{\headrulewidth}{0pt}

%\usepackage{caption}
% ---------------------------------------------------------------------------- %
\graphicspath{{./figuras/}}             % caminho das figuras (recomendável)
\frenchspacing                          % arruma o espaço: id est (i.e.) e exempli gratia (e.g.) 
\urlstyle{same}                         % URL com o mesmo estilo do texto e não mono-spaced
\makeindex                              % para o índice remissivo
\raggedbottom                           % para não permitir espaços extra no texto
\fontsize{60}{62}\usefont{OT1}{cmr}{m}{n}{\selectfont}
\cleardoublepage
\normalsize
%%Comment Out
\newcommand{\commentOut}[1]{}
% ---------------------------------------------------------------------------- %
% Opções de listing usados para o código fonte
% Ref: http://en.wikibooks.org/wiki/LaTeX/Packages/Listings
\lstset{ %
language=Java,                  % choose the language of the code
basicstyle=\footnotesize,       % the size of the fonts that are used for the code
numbers=left,                   % where to put the line-numbers
numberstyle=\footnotesize,      % the size of the fonts that are used for the line-numbers
stepnumber=1,                   % the step between two line-numbers. If it's 1 each line will be numbered
numbersep=5pt,                  % how far the line-numbers are from the code
showspaces=false,               % show spaces adding particular underscores
showstringspaces=false,         % underline spaces within strings
showtabs=false,                 % show tabs within strings adding particular underscores
frame=single,	                % adds a frame around the code
framerule=0.6pt,
tabsize=2,	                    % sets default tabsize to 2 spaces
captionpos=b,                   % sets the caption-position to bottom
breaklines=true,                % sets automatic line breaking
breakatwhitespace=false,        % sets if automatic breaks should only happen at whitespace
escapeinside={\%*}{*)},         % if you want to add a comment within your code
backgroundcolor=\color[rgb]{1.0,1.0,1.0}, % choose the background color.
rulecolor=\color[rgb]{0.8,0.8,0.8},
extendedchars=true,
xleftmargin=10pt,
xrightmargin=10pt,
framexleftmargin=10pt,
framexrightmargin=10pt
}

\lstdefinestyle{customc}{
  escapeinside={\%*}{*)},
  captionpos = b,
  belowcaptionskip=1\belowcaptionskip,
  breaklines=true,
  numbers=left,
  numbersep=5pt,
  numberstyle=\tiny\color{gray},
  language=C++,
  showstringspaces=false,
  basicstyle=\footnotesize\ttfamily,
  morekeywords={*,para,hasta,inicio,fin,repetir,veces},
  keywordstyle=\bfseries\color{green!40!black},
  commentstyle=\itshape\color{purple!40!black},
  identifierstyle=\color{blue},
  stringstyle=\color{orange},
  emph = {Union}, emphstyle = \color{purple},
  emph = {[2]Find}, emphstyle = {[2]\color{purple}}
}
\renewcommand*\lstlistingname{Algoritmo}
%----------------------------%
\hyphenation{per-so-nal res-pec-to}


% ---------------------------------------------------------------------------- %
% Corpo do texto
\parindent=2.5em
\begin{document}
\frontmatter 
% cabeçalho para as páginas das seções anteriores ao capítulo 1 (frontmatter)
\fancyhead[RO]{{\footnotesize\rightmark}\hspace{2em}\thepage}
\setcounter{tocdepth}{4}
\setcounter{secnumdepth}{4}
\fancyhead[LE]{\thepage\hspace{2em}\footnotesize{\leftmark}}
\fancyhead[RE,LO]{}
\fancyhead[RO]{{\footnotesize\rightmark}\hspace{2em}\thepage}
\onehalfspacing  % espaçamento

% ---------------------------------------------------------------------------- %
% CAPA
% Nota: O título para as dissertações/teses do IME-USP devem caber em um 
% orifício de 10,7cm de largura x 6,0cm de altura que há na capa fornecida pela SPG.
\thispagestyle{empty}

\begin{center}

\textsc{\large Universidad Nacional de San Antonio Abad del Cusco}\\\vspace*{0.04in}
\textsc{FACULTAD DE CIENCAS QUÍMICAS, FÍSICAS Y MATEMÁTICAS}\\
\vspace*{0.04in}
CARRERA PROFESIONAL DE INGENIERÍA INFORMÁTICA Y DE SISTEMAS \\

\captionsetup[figure]{labelformat=empty}
\begin{figure}[htb]
\begin{center}
\includegraphics[width=6cm]{UNSAAC}
\caption[]{}
\end{center}
\end{figure}

\newcommand{\topline}{
  \rule{164mm}{2mm}
  \vspace*{-0.23in}
  \hrule  
}
\newcommand{\downline}{
  \hrule  
  \vspace*{0.02in}
  \rule{164mm}{2mm}
}

\vspace*{-0.6in}
\textbf{TESIS}\\
\topline
\vspace*{0.1in}
\begin{large}
\textbf{``TITULO DEL TRABAJO''} \\
\end{large}
\vspace*{0.08in}
\downline
\vspace*{0.25in}

\begin{minipage}{\linewidth}
  \large
  \centering    
  \begin{minipage}{0.45\linewidth}
  \end{minipage}
  \hspace{0.28\linewidth}
  \begin{minipage}{0.7\linewidth}
    \begin{normalsize}
    Para optar al título profesional de:
    \vspace*{-0.1in}
    \\\textbf{INGENIERO INFORMÁTICO Y DE SISTEMAS}\\
    Presentado por: \vspace*{-0.1in}
    \\\textbf{AUTOR}\\
    Asesor: \vspace*{-0.1in}
    \\\textbf{ASESOR}\\
    Co-asesores: \vspace*{-0.1in}
    \\\textbf{CO-ASESOR 1} \vspace*{-0.1in}
    \\\textbf{CO-ASESOR 2}
    \end{normalsize}
  \end{minipage}
\end{minipage}

\vspace*{0.5in}
\textbf{CUSCO - PERÚ} \\
\textbf{[A\~NO]}
\end{center}

\pagenumbering{roman}     % começamos a numerar 

% ---------------------------------------------------------------------------- %
% Agradecimentos:
% Se o candidato não quer fazer agradecimentos, deve simplesmente eliminar esta página 
\chapter*{Dedicatoria}
El presente trabajo está dedicado a ...
%\chapter*{Agradecimientos}
% [edit]
% ---------------------------------------------------------------------------- %

%\clearpage
%Una imagen vale más que mil palabras, pero pesa mucho más...

\chapter*{Resumen}

\blindtext
\\

\noindent \textbf{Palabras clave:} keyword1, keyword2, keyword3.

% ---------------------------------------------------------------------------- %
% Abstract
\chapter*{Abstract}

\blindtext
\\

\noindent \textbf{Keywords:} keyword1, keyword2, keyword3.

\tableofcontents    % imprime o sumário

% ---------------------------------------------------------------------------- %
% Listas de figuras e tabelas criadas automaticamente
\renewcommand\listtablename{ÍNDICE DE TABLAS}
\renewcommand\listfigurename{ÍNDICE DE FIGURAS}
\listoftables
\listoffigures

% ---------------------------------------------------------------------------- %
% Capítulos do trabalho
\mainmatter

% cabeçalho para as páginas de todos os capítulos
\fancyhead[RE,LO]{\thesection}

%\singlespacing              % espaçamento simples
\onehalfspacing            % espaçamento um e meio

\input cap-introducion
\part{ASPECTOS GENERALES}
\input cap-aspectosgenerales

\part{MARCO TEÓRICO}
\input cap-marcoteorico
\input cap-metodologia

\part{DESARROLLO DEL ENFOQUE PROPUESTO}
\input cap-aspectosenfoque  
\input cap-experimentos
\input cap-resultados

\input cap-conclusiones
\input cap-trabajosfuturos

\nocite{*}
% ---------------------------------------------------------------------------- %
% Bibliografia
\renewcommand\bibname{BIBLIOGRAFÍA}
\backmatter \singlespacing   % espaçamento simples
\bibliographystyle{ieeetr} % citação bibliográfica textual
\bibliography{bibliografia}  % associado ao arquivo: 'bibliografia.bib'

%-------------------------------------------
%\include{ape-conjuntos}      % associado ao arquivo: 'ape-conjuntos.tex'
\renewcommand{\appendixname}{Anexo}
\renewcommand{\appendixtocname}{ANEXOS}
\renewcommand{\appendixpagename}{ANEXOS}
\renewcommand{\chaptermark}[1]{\markboth{\MakeUppercase{\appendixname\ \thechapter}} {\MakeUppercase{#1}} }
\fancyhead[RE,LO]{}
\appendix
\clearpage
\addappheadtotoc
\appendixpage

\chapter*{A. DETALLES TÉCNICOS}
\label{ape:laptop}
\addcontentsline{toc}{chapter}{\numberline{}B. DETALLES TÉCNICOS}
Detalles técnicos del equipo de cómputo usado en las fases de implementación y pruebas.

%\begin{table}[h!]
\begin{center}
\centering
\begin{tabular}{|p{6cm}|p{10cm}|}
\hline
\textbf{Nombre del producto} & \textbf{dv6-2177la}                                                                                           \\ \hline
Microprocesador						   & Tecnología VISION Ultimate de AMD con procesador AMD Turion II Ultra Dual-Core M620 para notebooks a 2,56 GHz \\ \hline
Caché del microprocesador    & 2 MB de caché de nivel 2                                                                                      \\ \hline
Memoria                      & 4 GB de memoria DDR3 a 1066 MHz                                                                               \\ \hline
Gráficos de video            & ATI Mobility Radeon HD 4650                                                                                   \\ \hline
\end{tabular}
\textit{Información del computador usado durante la fase de testing.}
\end{center}
%\caption[]{Información del computador usado durante la fase de \textit{testing}.}
%\label{ape:laptop:detalle}
%\end{table}



\end{document}
